\documentclass[conference]{IEEEtran}
\IEEEoverridecommandlockouts
% The preceding line is only needed to identify funding in the first footnote. If that is unneeded, please comment it out.
\usepackage{cite}
\usepackage{amsmath,amssymb,amsfonts}
\usepackage{algorithmic}
\usepackage{graphicx}
\usepackage{textcomp}
\usepackage{xcolor}
\def\BibTeX{{\rm B\kern-.05em{\sc i\kern-.025em b}\kern-.08em
    T\kern-.1667em\lower.7ex\hbox{E}\kern-.125emX}}
\begin{document}

\title{Internet of Things (IoT) :Applications in  Home and Building Automation 
	\\

\thanks{Seminar of  "Internet of  Things" , January~2022}
}

\author{\IEEEauthorblockN{Marwa Mohammed Nabawey Hassan}
\IEEEauthorblockA{\textit{Department of  Electronic Engineering } \\
\textit{Hamm-Lippstadt university of applied sciences}\\
marwa-mohammed-nabawey.hassan@stud.hshl.de}

}

\maketitle

\begin{abstract}
	
The shift from the digital revolution to the fourth Industrial revolution, named Industry 4.0, allows us to utilize modern intelligent technologies such as the Internet of things. That increases the automation process through homes and buildings, aimed to reduce the direct human intervention, which causes reshaping in our ways of life. In addition, The integration of  IoT technology in houses and buildings offered a variety of services and comfort by controlling and monitoring the building's lights, temperature, and humidity remotely according to the users' needs and profiles. Furthermore, home appliances have been affected by these developments, allowing the users' to connect them to the Internet to provide and exchange information. This seminar paper presents such IoT applications that can be considered a base of smart homes and buildings. 



\end{abstract}

\begin{IEEEkeywords}
IOT , RFID ,smart home ,smart buildings 
\end{IEEEkeywords}

\section{Introduction}


The British pioneer Kevin Ashton was the first to announce the term "Internet of Things" during his presentation for Procter and Gamble in 1999. Ashton, the Executive Director of Auto-ID Labs at the Massachusetts Institute of Technology (MIT), invented the idea by putting Radio Frequency Identification (RFID) tag on different items, allowing them to communicate with a radio receiver. He considered that such an amount of data collection could solve many problems in real life. 

Kevin Ashton thought that Radio Frequency Identification (RFID)  was essential for the new technology name Internet of Things.  After connecting many devices, the computer could manage and track them .proving this idea makes any physical device capable of being interconnected with either other devices, computers, or the internet. Technologies like RFID, short-range wireless communications, real-time localization, and sensor networks are becoming increasingly pervasive, making the IoT a reality\cite{discovery}. The "Internet of Things" paradigm aims at providing models and mechanisms enabling the creation of networks of "smart things" on a large scale using RFID, wireless sensor and actuator networks, and embedded devices distributed in the physical environment \cite{RFID}.

That leads us to define the computing concept that describes the idea of connecting physical objects or groups of them which have sensors to be able to identify themselves to other devices, allowing them to exchange data within the system over the internet or locally with the name  Internet of Things (IoT). 

In this seminar paper, the application domains of IoT will be described briefly in section two. Section three will focus on explaining the use of IoT in home applications considering a use case, "Smart Plug." Through section four-building automation using IoT will be discussed considering a use case, " Smart Building." Section five will be a short evaluation and suggestions to overcome the challenges of IoT in Home and Building Automation. By the end of the seminar paper, a short conclusion summarizing the topic will be considered. 

\section{IOT Applications }


Internet of Things Strategic Research Agenda (SRA) \cite{SRA} defined more domains to afford new technology such as smart energy, smart health, smart buildings, smart transport, smart living, and smart cities. According to the research, IoT applications are classified into 14 different domains: Smart Home, Smart City, Lifestyle, Retail, Agriculture, Smart Factory, Supply chain, Emergency, Health care, User interaction, Culture, and Tourism, Environment, and Energy. 


During a survey done by Atzori et al. \cite{azori} they categorized
IoT applications. In their study, they grouped IoT applications   into the  three main domains: 

\begin{itemize}
	\item Environmental domain
	\item Industrial domain
	\item Social domain
	
\end{itemize}



\subsection{IoT Application in Society domain }

The main aim of IoT applications in this domain is to provide a real-time delivery system and reliable communications method to communicate. This will lead to a better life for people and work, on the other hand, to develop society by offering new intelligent services and technology like  Smart Cities, Smart Farming, Smart Agriculture,  Home automation, and Smart Buildings.

By monitoring parking and free spaces availability in smart cities, monitoring sound and weather conditions in critical areas, observing the vehicles and pedestrian levels, and adaptive lighting in street lights, all these applications indicate how IoT plays a vital role in smart cities. 



\subsection{IoT Application in Enviornment domain}

The IoT Application in the Environment domain covered all activities aimed to protect, develop, and monitor natural resources. It includes  Recycling, Smart Environment, Smart Water, and Disaster Alerting. 

Within the agriculture field, added IoT applications great support by monitoring soil moisture allowing the option to maintain and control the number of vitamins transferred to the soil. Furthermore, The role of IoT in water management by the study of water suitability in rivers and the sea for agriculture and drinkable use, detection of a liquid presence outside tanks.

\subsection{IoT Application in Industial domain}

IoT applications in the Industrial domain consider all the activities related to financial, commercial transactions between companies, organizations, and other entities, which are related to manufacturing, logistics, banking, financial, governmental authorities. This application includes Retail, Logistics, Supply Chain Management, Automotive, Industrial,
Control, Aerospace, and Aviation. 

Using  IoT in Retail and Supply Chain management will include monitoring and controlling storage along the supply chain by tracking products for traceability purposes. Moreover,  In the shops, implementing IoT technology will help users in guidance in the shop according to the pre-done shopping list, offering fast payment options solution 
like automatically check-out using detection of a given product. 



\section{Home Automation}

 
Early home automation  \& now \& IOT home appliances 
METHODOLOGIES 
Basic Block Diagram of Home Automation......smart-home controller ?? Alexa ,siri?!
\\use case 
challanges 

\subsection{Smart Plugs}






\section{Building  Automation }

Early control systems \& today control and automation \\ IOT change \\

Building automation model
challanges 

\subsection{smart buldings}

\section{Evaluation \& Future work   }


\section{Conclusions  }



\begin{thebibliography}{00}
	
	\bibitem{discovery}Zaslavsky, A., \& Jayaraman, P. P. (2015). Discovery in the Internet of Things. Ubiquity, 2015, 1-10.
	
 
 \bibitem {RFID}D. Guinard, V. Trifa, and E. Wilde, “A resource oriented architecture for the web of things,” in Proceedings of the 2nd International Internet of Things Conference (IoT ’10), pp. 9–129,
 December 2010.
 
 \bibitem{SRA}Tarkoma, S., \& Katasonov, A. (2011). Internet of things strategic research agenda (IoT–SRA). Finnish Strategic Centre for Science, Technology, and Innovation: For Information and Communications (ICT) Services, Businesses, and Technologies, Finland.
 
 \bibitem{azori}Atzori, L., Iera, A., \& Morabito, G. (2010). The internet of things: A survey. Computer networks, 54(15), 2787-2805.
 

\end{thebibliography}


\end{document}
