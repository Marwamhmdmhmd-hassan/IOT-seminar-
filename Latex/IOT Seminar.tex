\documentclass[conference]{IEEEtran}
\IEEEoverridecommandlockouts
% The preceding line is only needed to identify funding in the first footnote. If that is unneeded, please comment it out.
\usepackage{cite}
\usepackage{amsmath,amssymb,amsfonts}
\usepackage{algorithmic}
\usepackage{graphicx}
\usepackage{textcomp}
\usepackage{xcolor}
\def\BibTeX{{\rm B\kern-.05em{\sc i\kern-.025em b}\kern-.08em
    T\kern-.1667em\lower.7ex\hbox{E}\kern-.125emX}}
\begin{document}

\title{Internet of Things (IOT) :Applications in  Home and Building Automation 
	\\

\thanks{Seminar of  "Internet of  Things" , January~2022}
}

\author{\IEEEauthorblockN{Marwa Mohammed Nabawey Hassan}
\IEEEauthorblockA{\textit{Department of  Electronic Engineering } \\
\textit{Hamm-Lippstadt university of applied sciences}\\
marwa-mohammed-nabawey.hassan@stud.hshl.de}

}

\maketitle

\begin{abstract}
	
The shift from the digital revolution to the fourth Industrial revolution, named Industry 4.0, allows us to utilize modern intelligent technologies such as the Internet of things. That increases the automation process through homes and buildings, aimed to reduce the direct human intervention, which causes reshaping in our ways of life. In addition, The integration of  IoT technology in houses and buildings offered a variety of services and comfort by controlling and monitoring the building's lights, temperature, and humidity remotely according to the users' needs and profiles. Furthermore, home appliances have been affected by these developments, allowing the users' to connect them to the Internet to provide and exchange information. This seminar paper presents such IoT applications that can be considered a base of smart homes and buildings. 



\end{abstract}

\begin{IEEEkeywords}
IOT , smart home ,smart buildings 
\end{IEEEkeywords}

\section{Introduction}


\section{IOT Applications }

Application Domains
\\
1-Society  \\
2-Enviornment \\
3-Industial \\

\cite{AppOverview}
\cite{Domain}


\section{Home Automation}
 
Early home automation  \& now \& IOT home appliances 
METHODOLOGIES 
Basic Block Diagram of Home Automation......smart-home controller ?? Alexa ,siri?!
\\use case 

\subsection{Smart Plugs}

A smart home plug is an electric device that can be plugged
into an ordinary outlet. It provides outlets for other electronic
devices, e.g., lamps and fans. It is often designed to connect to
the wireless home network so that a user can install an app on
her smart device, e.g., smartphone, and control the electronic
device plugged into the smart plug over the Internet.

\section{Building  Automation }

Early control systems \& today control and automation \\ IOT change \\

Building automation model

\subsection{smart buldings}

\section{Conclusions and Future Work }



\begin{thebibliography}{00}
	
	
\bibitem{AppOverview} P. V. Dudhe, N. V. Kadam, R. M. Hushangabade and M. S. Deshmukh, "Internet of Things (IOT): An overview and its applications," 2017 International Conference on Energy, Communication, Data Analytics and Soft Computing (ICECDS), 2017, pp. 2650-2653, doi: 10.1109/ICECDS.2017.8389935..
 \bibitem{Domain} Porkodi, R.,  Bhuvaneswari, V. (2014, March). The internet of things (IOT) applications and communication enabling technology standards: An overview. In 2014 International conference on intelligent computing applications (pp. 324-329). IEEE.

\end{thebibliography}


\end{document}
